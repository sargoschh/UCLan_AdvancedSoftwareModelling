% Main Content Chapters

%========================================
% CHAPTER 1: INTRODUCTION
%========================================
\chapter{Introduction}

Purpose: Briefly introduce the real-time temperature regulation unit being developed for smart laboratory equipment.

Scope: Mention that the report covers the formal specification, UML modeling, and concurrent design required to meet high-integrity safety standards.
\newline

%========================================
% CHAPTER 2: System Requirements and Decomposition
%========================================
\chapter{System Requirements and Decomposition}

Functional Requirements: List the core operations, such as continuous monitoring of multiple sensors and activating cooling/heating components.

Non-Functional Requirements: Detail the constraints, including ±0.1C accuracy and the 1-second response time limit.

Subsystem Architecture: Describe the modular breakdown into SensorInput, TemperatureController, ActuatorControl, and SystemMonitoring.

Use Case Modeling: Present a UML use case diagram showing actors (e.g., Environment, System Operator) and their interactions.
\newline

%========================================
% CHAPTER 3: UML Structural and Behavioural Modelling
%========================================
\chapter{UML Structural and Behavioural Modelling}

Class Diagram: Provide a structural view of the system's classes, their relationships, and multiplicities.

Formal Constraints (OCL): Document the Object Constraint Language (OCL) expressions for invariants, preconditions, and postconditions.

Dynamic Behaviour: Include sequence or activity diagrams to model the control loop and sensor reading process.
\newline

%========================================
% CHAPTER 4: Formal Specification and Design by Contract (DbC)
%========================================
\chapter{Formal Specification and Design by Contract (DbC)}

Contract Specification: Provide the formal JML (Java Modeling Language) or annotated Java code for the TemperatureController.

Verification Logic: Explain how Design by Contract (DbC) principles ensure correctness by preventing invalid states and software faults.

Testing Support: Discuss how these specifications assist in the verification and testing process.
\newline

%========================================
% CHAPTER 5: Concurrency and Real-Time Behaviour
%========================================
\chapter{Concurrency and Real-Time Behaviour}

Thread Management: Model the concurrent components, such as multiple sensor threads accessing shared data.

Synchronization Strategy: Identify potential race conditions or deadlocks and explain your use of mutexes, semaphores, or monitors to solve them.

Implementation: Provide Java code snippets demonstrating thread synchronization or real-time scheduling.
\newline

%========================================
% CHAPTER 6: Conclusion and Evaluation
%========================================
\chapter{Conclusion and Evaluation}

Summary: Summarize how the formal models and specifications support the reliability and performance goals of the system.

Technique Evaluation: Briefly evaluate the software development techniques used (UML, OCL, JML, and Java concurrency) in terms of their effectiveness for high-integrity systems.
\newline